Le probl\`eme que nous nous proposons d'aborder est la gestion de variables exploit\'ees
par plusieurs t\^aches, et en particulier la coh\'erence des acc\`es au contenu des 
emplacement m\'emoire associ\'e. 

Un probl\`eme classique consiste en deux t\^aches qui tentent de manipuler une variable
selon un processus complexe~: au d\'ebut de la t\^ache la valeur de la variable est lue,
le traitement s'op\`ere pendant un certain temps, et finalement le r\'esultat des 
calculs est stock\'e dans cette m\^eme variable. La catastrophe survient si la seconde
t\^ache engage une manipulation sur cette m\^eme variable alors que le calcul est en cours.

Ce probl\`eme est classique dans les syst\`emes embarqu\'es o\`u plusieurs t\^aches manipulent
une m\^eme variable, mais aussi dans les interfaces graphiques o\`u les diverses fen\^etres
sont g\'er\'ees par des processus ind\'ependant qui ont toutes les chances de manipuler des
donn\'ees communes.

Une des t\^aches des syst\`emes d'exploitation consiste \`a fournir des m\'ecanismes de
protection de variables au cours de leur manipulation pour interdire tout acc\`es par une
seconde t\^ache. Un bout de code est donc prot\'eg\'e d'acc\`es par une autre t\^ache au
moyen de la m\'ethode des MUTEX (Mutually exclusive). Un mutex est pos\'e en d\'ebut de
segment de code \`a prot\'eger, et toute autre t\^ache qui tentera d'acc\'eder \`a ce bout
de code alors qu'il est d\'ej\`a en cours d'ex\'ecution se verra bloqu\'ee jusqu'\`a ce 
que le mutex soit rel\^ach\'e. Si le calcul sur la variable est ainsi prot\'eg\'e par mutex,
nous garantissons que une seule t\^ache manipule la donn\'ee \`a chaque instant.

En pratique, une cons\'equence du mutex et de la mise en attente de la seconde t\^ache qui
tente d'intervenir sur le bout de code prot\'eg\'e est de r\'eorganiser la s\'equence des
calculs. Nous verrons donc que des modifications {\em a priori} anodines sur le code se
traduit par une r\'eorganisation drastique de la s\'equence d'\'ex\'ecution.

Au lieu de lancer un calcul complexe, nous simulerons une activit\'e longue sur une 
variable par son affichage au travers d'un bus asynchrone (RS232), une t\^ache tr\`es
lente compar\'ee \`a la vitesse de cadencement d'un processeur.
